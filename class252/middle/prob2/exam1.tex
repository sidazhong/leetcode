\documentclass[addpoints]{exam}

\usepackage{listings}
\usepackage{url}
\usepackage{amsmath}

% Code details
\lstset{
  language=Haskell,
  tabsize = 4,
  escapechar=@,
  basicstyle=\ttfamily,
  commentstyle=\tt,
}

% Header and footer details
\pagestyle{headandfoot}
\header{CS 252, Takehome Exam 1}{}{Name: \bigblank}
%\headrule
\footer{}{Page \thepage\ of \numpages}{}

\setlength\answerskip{0ex}

% Spacing and formatting, especially for blank lines.
\newcommand{\code}[1]{{\tt #1}}
\newcommand{\blank}{\makebox[1in]{\hrulefill}}
\newcommand{\midblank}{\makebox[.7in]{\hrulefill}}
\newcommand{\bigblank}{\makebox[2in]{\hrulefill}}
\newcommand{\hugeblank}{\makebox[3in]{\hrulefill}}
\newcommand{\ansline}{\hfill\makebox[1in][r]{\hrulefill}}
\newcommand{\anslinelong}{\hfill\makebox[2in][r]{\hrulefill}}

\begin{document}




\begin{questions}

  \question[15]
  Consider the following language:

  % Commands for formatting figure
\newcommand{\mydefhead}[2]{\multicolumn{2}{l}{{#1}}&\mbox{\emph{#2}}\\}
\newcommand{\mydefcase}[2]{\qquad\qquad\qquad\qquad\qquad\qquad& #1 &\mbox{#2}\\}

% Commands for language format
\newcommand{\setexp}[2]{\mbox{\tt set}~#1~#2}
\newcommand{\readexp}[1]{\mbox{\tt read}~#1}
\newcommand{\loopexp}[2]{\mbox{\tt loop}~#1~#2}
\newcommand{\incexp}[1]{\mbox{\tt inc}~{#1}}
\newcommand{\testexp}[2]{\mbox{\tt test}~{#1}~{#2}}
\newcommand{\thenexp}[2]{#1~\mbox{\tt then}~{#2}}
\newcommand{\crashexp}{\mbox{\tt crash}}

\newcommand{\rel}[1]{ \mbox{\sc [#1]} }

\newcommand{\sstep}[2]{{#1} \rightarrow {#2}}
\newcommand{\sstepStore}[4]{{#1},{#2} \rightarrow {#3},{#4}}
\newcommand{\bstep}[2]{{#1} \Downarrow {#2}}
\newcommand{\bstepStore}[4]{{#1},{#2} \Downarrow {#3},{#4}}

\newcommand{\ssrule}[3]{
  \rel{#1} &
  \frac{\strut\begin{array}{@{}c@{}} #2 \end{array}}
       {\strut\begin{array}{@{}c@{}} #3 \end{array}}
   \\~\\
}

\[
  \begin{array}{ll@{\qquad\qquad\qquad}r}
  \mydefhead{e ::=}{Expressions}
    \mydefcase{n}{number (integer)}
    \mydefcase{\setexp{x}{e}}{set variable}
    \mydefcase{\readexp{x}}{read variable}
    \mydefcase{\incexp{x}}{increment operator}
    \mydefcase{\thenexp{e}{e}}{then expression}
    \mydefcase{\testexp{e}{e}}{test expression}
    \mydefcase{\loopexp{e}{e}}{loop}
    \mydefcase{\crashexp}{crash expression}
  \\
  \mydefhead{x ::=\qquad\qquad\qquad\qquad}{Variables}
\end{array}
\]

  The big-step operational semantics relation evaluates an expression to a {\bf number}.
  \[\large
    \bstepStore{e}{\sigma}{n}{\sigma'}
  \]




% Commands for language format
\newcommand{\rval}{\mbox{\tt red}}
\newcommand{\gval}{\mbox{\tt green}}
\newcommand{\bval}{\mbox{\tt blue}}
\newcommand{\rot}[1]{\mbox{\tt rotate}~{#1}}
\newcommand{\ife}[3]{\mbox{\tt if}~({#1})~{\tt then}~({#2})~{\tt else}~({#3})}

Consider the following language and big-step operational semantics:
\[
  \begin{array}{ll@{\qquad\qquad\qquad}r}
  \mydefhead{e ::=}{Expressions}
  \mydefcase{v}{value}
  \mydefcase{\ife e e e}{if expression}
  \mydefcase{\rot e}{rotate expression}
  \\
  \mydefhead{v ::=\qquad\qquad\qquad\qquad}{Values}
  \mydefcase{\rval}{red}
  \mydefcase{\gval}{green}
  \mydefcase{\bval}{blue}
\end{array}
\]

\[
  %\begin{array}{c@{\qquad\qquad}c}
  \begin{array}{r@{\qquad}c}

    \ssrule{if-red}{
      \bstep{e_1}{v_1} \\
      v_1 = \rval \\
      \bstep{e_3}{v_3}
    }{
      \bstep{\ife{e_1}{e_2}{e_3}}{v_3}
    }

    \ssrule{if-other}{
      \bstep{e_1}{v_1} \\
      v_1 \neq \rval \\
      \bstep{e_2}{v_2}
    }{
      \bstep{\ife{e_1}{e_2}{e_3}}{v_2}
    }

  %\end{array}
  %&
  %\begin{array}{r@{\qquad}c}


    \ssrule{rot-red}{
     \bstep{e_1}{v_1} \\
      \bstep{v_1}{\rval}
    }{
      \bstep{\rot e_1}{\gval}
    }

    \ssrule{rot-green}{
      \bstep{e_1}{v_1} \\
      \bstep{v_1}{\gval}
    }{
      \bstep{\rot e_1}{\bval}
    }

    \ssrule{rot-blue}{
      \bstep{e_1}{v_1} \\
      \bstep{v_1}{\bval}
    }{
      \bstep{\rot e_1}{\rval}
    }
    
  \end{array}
  %\end{array}
\]

\end{questions}
\end{document}

