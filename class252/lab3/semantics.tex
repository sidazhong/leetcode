\documentclass{article}

\usepackage{fullpage}
%\usepackage{citesort}
%\usepackage{url}
%\usepackage{listings}
\usepackage{color}


% Commands for formatting figures
\newcommand{\mydefhead}[2]{\multicolumn{2}{l}{{#1}}&\mbox{\emph{#2}}\\}
\newcommand{\mydefcase}[2]{\qquad\qquad& #1 &\mbox{#2}\\}

% Commands for language expressions and values
\newcommand{\true}{\mbox{\tt true}}
\newcommand{\false}{\mbox{\tt false}}\title{Operational Semantics for Bool* Language}
\newcommand{\ife}[3]{\mbox{\tt if}~{#1}~\mbox{\tt then}~{#2}~\mbox{\tt else}~{#3}}
\newcommand{\suc}[1]{\mbox{\tt succ}~{#1}}
\newcommand{\prd}[1]{\mbox{\tt pred}~{#1}}

% Big-step evaluation relation.
% #1 is the expression.
% #2 is the resulting value.
\newcommand{\bstep}[2]{{#1} \Downarrow {#2}}

% Command for formatting the name of the evaluation relation.
\newcommand{\rel}[1]{ \mbox{\sc [#1]} }

% Format for a big-step evaluation rule.
% #1 is the name of the rule.
% #2 are the premises.  Leave blank if there are none.
% #3 is the conclusion.
\newcommand{\bsrule}[3]{
  \rel{#1} &
  \frac{\strut\begin{array}{@{}c@{}} #2 \end{array}}
       {\strut\begin{array}{@{}c@{}} #3 \end{array}}
   \\~\\
}

\author{
  Thomas H. Austin \\
  San Jos\'{e} State University \\
  thomas.austin@sjsu.edu
  }

\begin{document}
\maketitle

\begin{abstract}
In this paper, we will provide a review of the big-step operational semantics
for the Bool* language we discussed in class.
\end{abstract}

%%%%
%\section{Introduction}

Bool* is a very minimal language that allows us to experiment with operational semantics.

First, we define the valid expressions in our language.
These expressions dictate the possibilities of what expressions
we may have in our source programs.
(Note that other language might also have {\em statements}.
Statements might not evaluate to a value;
expressions always will.)

\begin{figure}
\caption{The Bool* language}
\label{fig:lang}
\[
  \begin{array}{ll@{\qquad\qquad}l}
  \mydefhead{e ::=\qquad\qquad\qquad\qquad}{Expressions}
  \mydefcase{\true}{true value}
  \mydefcase{\false}{false value}
  \mydefcase{\ife e e e}{conditional expressions}
  \\
  \mydefhead{v ::=\qquad\qquad\qquad\qquad}{Values}
  \mydefcase{\true}{true value}
  \mydefcase{\false}{false value}
\end{array}
\]
\end{figure}

Figure~\ref{fig:lang} shows the list of expressions and values for the Bool* language.
Expressions can be the value $\true$, the value $\false$,
or the conditional expression $\ife e e e$.
Note that conditional expressions have a recursive structure,
with 3 sub-expressions.

After evaluating a program, we should be able to produce a value in this language.
(In other languages, we might hit a bad situation and need to crash instead;
that won't happen in this language.)
The valid values for Bool* are $\true$ and $\false$.

With our expressions and values defined,
we can now specify the semantics for our language.
To do so, we will use the following big-step evaluation relation:

\[
  \bstep{e}{v}
\]

The above line should be read as ``the expression $e$ evaluates to the value $v$''.

\begin{figure}
\caption{Big-step semantics for Bool*}
\label{fig:bigstep}
  {\bf Evaluation Rules:~~~ \fbox{$\bstep{e}{v}$}} \\
\[
\begin{array}{r@{\qquad\qquad}c}
\bsrule{B-Value}{}{
  \bstep{v}{v}
}
\bsrule{B-IfTrue}{
  \bstep{e_1}{\true} \\
  \bstep{e_2}{v}
}{
  \bstep{\ife{e_1}{e_2}{e_3}}{v}
}
\bsrule{B-IfFalse}{
  \bstep{e_1}{\false} \\
  \bstep{e_3}{v}
}{
  \bstep{\ife{e_1}{e_2}{e_3}}{v}
}
\end{array}
\]
\end{figure}



Figure~\ref{fig:bigstep} shows the big-step evaluation rules for the Bool* language.
Of course, there are additional possible rules.

The \rel{B-Value} rule applies when the expression (to the left of ``$\Downarrow$'')
is also a value, as defined in Figure~\ref{fig:lang}.
There are no premises for this rule (above the line),
meaning that it is an {\em axiom}.
This rule states that a value evaluates to itself,
so that \true{} evaluates to \true{} and \false{} evaluates to \false{}.

Two different rules are needed for handling conditional expressions.
Which rule applies depends on the premises.

For the \rel{B-IfTrue} rule,
the premise states that $e_1$ evaluates to \true{}
and $e_2$ evaluates to some value $v$.
If the premise holds, then the result of evaluating the expression is the value $v$.
The structure of \rel{B-IfFalse} is similar.


\section*{Exercise}

Let's extend the language with new features.
Our new language will be called BoolNum*.
The valid expressions and values are shown below:

\begin{figure}
\caption{The BoolNum* language}
\label{fig:lang2}
\[
  \begin{array}{ll@{\qquad\qquad}l}
  \mydefhead{e ::=\qquad\qquad\qquad\qquad}{Expressions}
  \mydefcase{\true}{true value}
  \mydefcase{\false}{false value}
  \mydefcase{n}{natural numbers}
  \mydefcase{\ife e e e}{conditional expressions}
  \mydefcase{\suc e}{successor}
  \mydefcase{\prd e}{predecessor}
  \\
  \mydefhead{v ::=\qquad\qquad\qquad\qquad}{Values}
  \mydefcase{\true}{true value}
  \mydefcase{\false}{false value}
  \mydefcase{n}{natural numbers}
\end{array}
\]
\[
\begin{array}{r@{\qquad\qquad}c}
\bsrule{BN-Value}{}{
  \bstep{v}{v}
}
\bsrule{BN-IfTrue}{
  \bstep{e_1}{\true} \\
  \bstep{e_2}{v}
}{
  \bstep{\ife{e_1}{e_2}{e_3}}{v}
}
\bsrule{BN-IfFalse}{
  \bstep{e_1}{\false} \\
  \bstep{e_3}{v}
}{
  \bstep{\ife{e_1}{e_2}{e_3}}{v}
}
\bsrule{BN-SUCC}{
  \bstep{e}{n} \\
  n = n'+1 
}{
  \bstep{\suc{e}}{n'}
}
\bsrule{BN-PRED}{
  \bstep{e}{n} \\
  n = n'-1  \\
  n > 0
}{
  \bstep{\prd{e}}{n'}
}
\bsrule{BN-PRED-IF-0}{}{
  \bstep{\prd{0}}{0}
}
\end{array}
\]
\end{figure}

In addition to numbers, our expanded language has the ability to increment a number ($\suc e$)
and decrement a number ($\prd e$).

Using LaTeX, write the evaluation rules for these additional constructs.
If necessary, revise any existing rules.

Then, add these new constructs to the Haskell interpreter that we have been working with.
Note that there are a few reasonable choices on how these features should work.
Whatever you choose is fine, as long as your semantics match your interpreter.


\end{document}

